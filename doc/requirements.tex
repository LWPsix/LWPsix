\documentclass{article}
\newcommand{\LWP}{\emph{LWP(original)}}
\newcommand{\LWPsix}{\emph{LWPsix}}
\begin{document}

\part*{Goals}
    \LWPsix{} is the Perl 6 analog of Perl's \LWP{}; as such, the considerations
    behind \LWPsix{} and the differences from its predecessor are very similar
    to the evolution from Perl to Perl 6.
    
    Larry Wall said of Perl 6 ``we decided it would be better to fix the
    language than to fix the user''; in \LWPsix{}, we aim to identify the parts
    of the \LWP{} API that have been used in similar ways in many applications,
    and reorient the library such that the most obvious interfaces are those
    that most directly apply to the most common usage patterns -- taking the
    parts that users frequently implement and the helpers that they most
    frequently employ, and integrating them into the library front and center.
    We seek to make the easy things easier, while taking care to keep the hard
    things accessible. We want to make the highest-level interfaces that are in
    frequent use the most obvious interfaces, while still offering more raw
    access -- which we would design to be accessed primarily by subclassers.

    Also we'll be Perl 6-idiomatic, so we will have differences mirroring the
    differences between the language styles. ``Real'' OOP, roles, rules,
    junctions, etc will affect things.

\part*{\LWPsix{} usage}
    \section*{Overview}
        At the highest level, the interface follows the same paradigm as \LWP{}:
        \begin{itemize}
            \item create a \texttt{Request}
            \item give it to a \texttt{UserAgent}
            \item get a \texttt{Response}
        \end{itemize}
        The service is stateless; all the information required to make a request is
        specified in that \texttt{Request} object.

% this is a bunch of stuff from the LWP perldoc that seems like it's what LWP
% consider their ``requirements'', so our own requirement set should probably be
% informed by these:
\part*{\LWP{} is}
    ``a simple and consistent \ldots API to the World-Wide Web''

    ``The main focus of the library is to provide classes and functions that
    allow you to write WWW clients.''

    ``The library also contain modules that are of more general use and even
    classes that help you implement simple HTTP servers.''

    ``Most modules in this library provide an object oriented API. The user
    agent, requests sent and responses received from the WWW server are all
    represented by objects. This makes a simple and powerful interface to these
    services. The interface is easy to exntend and customize for your own needs.

    ``Contains various reusable components (modules) that can be used separately
    or together.''

    ``Provides an object oriented model of HTTP-style communication. Within this
    framework we currently support access to http, https, gopher, ftp, news,
    file, and mailto resources.''

    ``Provides a full object oriented interface or a very simple procedural
    interface.''

    ``Supports the basic and digest authorization schemes.''

    ``Supports transparent redirect handling.''

    ``Supports access through proxy servers.''

    ``Provides parser for robots.txt files and a framework for constructing
    robots.''

    ``Supports parsing of HTML forms.''

    ``Implements HTTP content negotiation algorithm that can be used both in
    protocol modules and in server scripts (like CGI scripts).''

    ``Supports HTTP cookies.''

    ``Some simple command line clients, for instance ``lwp-request'' and
    ``lwp-download''.''


\end{document}
